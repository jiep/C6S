\documentclass[12pt,a4paper,twoside,openright,titlepage,final]{article}
\usepackage{fontspec}
\usepackage{amsmath}
\usepackage{amsfonts}
\usepackage{amssymb}
\usepackage{makeidx}
\usepackage{graphicx}
\usepackage[hidelinks,unicode=true]{hyperref}
\usepackage[spanish,es-nodecimaldot,es-lcroman,es-tabla,es-noshorthands]{babel}
\usepackage[left=3cm,right=2cm, bottom=4cm]{geometry}
\usepackage{natbib}
\usepackage{microtype}
\usepackage{ifdraft}
\usepackage{verbatim}
\usepackage[obeyDraft]{todonotes}
\ifdraft{
	\usepackage{draftwatermark}
	\SetWatermarkText{BORRADOR}
	\SetWatermarkScale{0.7}
	\SetWatermarkColor{red}
}{}
\usepackage{booktabs}
\usepackage{longtable}
\usepackage{calc}
\usepackage{array}
\usepackage{caption}
\usepackage{subfigure}
\usepackage{footnote}
\usepackage{url}
\setsansfont[Ligatures=TeX]{texgyreadventor}
\setmainfont[Ligatures=TeX]{texgyrepagella}

\input{portada}

\author{José Ignacio Escribano}

\title{}

\setlength{\parindent}{0pt}

\begin{document}

\pagenumbering{alph}
\setcounter{page}{1}

\portada{Ejemplo Práctico}{Calidad Seis Sigma}{Proceso de producción de helicópteros}{José Ignacio Escribano}{Móstoles}

\listoffigures
\thispagestyle{empty}
\newpage

\listoftables
\thispagestyle{empty}
\newpage

\tableofcontents
\thispagestyle{empty}
\newpage


\pagenumbering{arabic}
\setcounter{page}{1}

\section{Definición}

La empresa Parasafe S.A. se dedica al diseño, producción y venta de helicópteros de papel.\\

Estos helicópteros se utilizan para realizar estudios de aerodinámica en diseño de túneles de viento, separadores ciclónicos y sistemas de ventilación especiales.\\

El proceso de fabricación de estos helicópteros consta de cuatro etapas básicas: el aprovisionamiento de materia prima, el montaje, la prueba de vuelo y el etiquetado final previo al envío al cliente.\\

El montaje consta de dos subprocesos: el corte del papel y el pegado del mismo. El corte consiste en separar el borde del patrón y realizar los cortes señalados; el pegado consiste en unir los bordes del cuerpo con cinta adhesiva corriente. El proceso se hace de forma manual.\\

La prueba de vuelo consiste en lanzar cada helicóptero, en posición vertical, desde una altura de 2 metros, y midiendo el tiempo que tarda en caer al suelo. Se considera que el helicóptero pasa la prueba si el tiempo de vuelo es mayor o igual a 1 segundo. La prueba se realiza con un cronógrafo manual, capaz de medir 1/100 segundos.\\

El equipo de producción de Parasafe S.A. consta de 10 personas, que trabajan un único turno de 8 horas/día. El reparto de empleados entre las diferentes tareas es el siguiente:

\begin{itemize}
	\item Etapa de inspección: 1 personas más medio turno de otra
	\item Etapa de corte: 2 personas más medio turno de otra
	\item Etapa de pegado: 1 persona más medio turno de otra
	\item Etapa de la prueba de vuelo: 3 personas
	\item Etapa de etiquetado: 1 persona más medio turno de otra
\end{itemize}

El tiempo que se emplea en la fabricación de un helicóptero se desglosa a continuación:

\begin{itemize}
	\item Etapa de inspección: 35 segundos/unidad
	\item Etapa de corte: 55 segundos/unidad
	\item Etapa de pegado: 35 segundos/unidad
	\item Etapa de la prueba de vuelo: 55 segundos/unidad
	\item Etapa de etiquetado: 35 segundos/unidad
\end{itemize}

Los costes de producción se describen a continuación:

\begin{enumerate}
	\item Costes fijos
	\begin{itemize}
		\item Salario de los empleados
		\begin{itemize}
			\item 1200 €/mes por operario
			\item 1900 €/mes por técnico
		\end{itemize}
		\item Alquiler y gastos de mantenimiento: 4000 €/mes 
	\end{itemize}
	
	\item Coste del papel (tamaño DIN A4)
	\begin{itemize}
		\item Suministrador A (buena calidad): 0.8 €/hoja
		\item Suministrador B (mala calidad): 0.6 €/hoja
	\end{itemize}
	
	\item Coste de inspección: 0.55 €/unidad
	\item Coste del corte: 1.5 €/unidad
	\item Coste del pegado: 0.45 €/unidad
	\item Coste de la prueba de vuelo: 1.5 €/unidad
	\item Coste del etiquetado: 0.55 €/unidad
\end{enumerate}

El precio de venta actual es de 6 €/unidad.\\

Teniendo en cuenta los tiempos diarios dedicados a cada tarea (sumando a todos los empleados) y el tiempo de realización de cada tarea en cada unidad se tiene que las unidades que se pueden realizar en cada tarea son las siguientes:

\begin{table}[htbp!]
	\centering
	\caption{Número de unidades diarias por tareas}
	\label{tbl:tiempos}
	\begin{tabular}{@{}cccc@{}}
		\toprule
		Tarea        & \begin{tabular}[c]{@{}c@{}}Tiempo diario\\ (en horas)\end{tabular} & \begin{tabular}[c]{@{}c@{}}Tiempo en realizar tarea\\ por unidad (en segundos)\end{tabular} & \begin{tabular}[c]{@{}c@{}}Número de \\ unidades diarias\end{tabular} \\ \midrule
		Inspección   & 12                                                                 & 35                                                                                          & 1\,234                                                                  \\
		Corte        & 20                                                                 & 55                                                                                          & 1\,309                                                                  \\
		Pegado       & 12                                                                 & 35                                                                                          & 1\,234                                                                  \\
		Prueba vuelo & 24                                                                 & 55                                                                                          & 1\,570                                                                  \\
		Etiquetado   & 12                                                                 & 35                                                                                          & 1\,234                                                                  \\ \bottomrule
	\end{tabular}
\end{table} 

A la vista de la Tabla~\ref{tbl:tiempos} tenemos una producción de 1\,234 unidades diarias, que al mes son 37\,028 unidades.\\

Estas 37\,028 suponen unos ingresos de 222\,168 € mensuales.\\

Los costes mensuales asociados son de 220\,400 €.\\

Por tanto, el beneficio mensual es de 1\,768.2 €.\\

Los objetivos generales del proyecto son los siguientes:

\begin{itemize}
	\item El tiempo de vuelo debe ser mayor de 1 segundo. Sólo 1 de cada 2000 helicópteros podrá no cumplir este requisito de calidad.
	\item El coste de producción debe ser mínimo
\end{itemize} 



\section{Medida}

\section{Análisis gráfico}

\subsection{Análisis numérico}

\section{Mejora}

\subsection{Optimización del diseño y del proceso de fabricación}

\section{Control}

\section{Conclusiones}





\end{document}